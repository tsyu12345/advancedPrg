\documentclass[dvipdfmx]{jsarticle}
\usepackage[T1]{fontenc}
\usepackage[dvipdfmx]{hyperref}
\usepackage{lmodern}
\usepackage{latexsym}
\usepackage{amsfonts}
\usepackage{amssymb}
\usepackage{mathtools}
\usepackage{amsthm}
\usepackage{multirow}
\usepackage{graphicx}
\usepackage{wrapfig}
\usepackage{here}
\usepackage{float}
\usepackage{ascmac}
\usepackage{url}

\title{Javaを用いたデザインパターンコーディングの必要性}
\author{文理学部情報科学科\\5419045 高林 秀}
\date{\today}

\begin{document}

\maketitle

\begin{abstract}
本稿では、今年度発展プログラミングの課題研究として「デザインパターン」に準拠したコーディングスタイルの必要性について、実際に自身でコーディングを行い、論ずる。本演習にはJavaを利用した。
\end{abstract}

\section{目的}
本稿は今年度発展プログラミングの課題研究として、「デザインパターン」に準拠したコーディングスタイルの必要性について論ずることを目的とする。本稿前半では、デザインパターンの利点・効果・必要性などについて説明し、後半では、実際にコーディングを行う。その際、デザインパターンを使用する前のコードとデザインパターンを使用した際のコードを比較・考察していく。
\section{デザインパターンとはなにか}

\section{実際のコーディングでデザインパターンを利用する}
\subsection{課題説明}
\paragraph{問題} 任意のデザインパターンを1つ選択し、そのパターンを用いる前と用いた後のコード両方を作成して示し、そのパターンを用いたことの効果を説明せよ。
\begin{itemize}
  \item 選択したデザインパターン:
\end{itemize}

\subsubsection{演習環境}
今回の演習は仮想マシン上でJavaを使用し行った。下記に演習時の環境を示す。
\begin{itemize}
  \item ホストOS:Window10 Home 20H2
  \item 仮想OS:Ubuntu 20.04.2 LTS
  \item CPU:Intel(R)Core(TM)i7-9700K @ 3.6GHz
  \item GPU:Nvidia Geforce RTX2070 OC @ 8GB
  \item ホストRAM:16GB
  \item 仮想RAM:4GB
  \item 使用言語:Java
  \begin{itemize}
    \item バージョン情報は下記に示す。
    \begin{verbatim}
      openjdk version "11.0.11" 2021-04-20
      OpenJDK Runtime Environment (build 11.0.11+9-Ubuntu-0ubuntu2.20.04)
      OpenJDK 64-Bit Server VM (build 11.0.11+9-Ubuntu-0ubuntu2.20.04, mixed mode, sharing)
    \end{verbatim}
  \end{itemize}
\end{itemize}

\subsection{制作物}
\subsubsection{デザインパターン前のソースコード}
\subsubsection{デザインパターン後のソースコード}
\subsection{考察}
\section{まとめ}
\section{巻末資料}

\end{document}
