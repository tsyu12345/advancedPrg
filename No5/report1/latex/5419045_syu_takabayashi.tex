\documentclass[dvipdfmx]{jsarticle}
\usepackage[T1]{fontenc}
\usepackage[dvipdfmx]{hyperref}
\usepackage{lmodern}
\usepackage{latexsym}
\usepackage{amsfonts}
\usepackage{amssymb}
\usepackage{mathtools}
\usepackage{amsthm}
\usepackage{multirow}
\usepackage{graphicx}
\usepackage{wrapfig}
\usepackage{here}
\usepackage{float}
\usepackage{ascmac}
\usepackage{url}

\title{Javaを用いたデザインパターンコーディングの必要性}
\author{文理学部情報科学科\\5419045 高林 秀}
\date{\today}

\begin{document}

\maketitle

\begin{abstract}
本稿では、今年度発展プログラミングの課題研究として「デザインパターン」に準拠したコーディングスタイルの必要性について、実際に自身でコーディングを行い、論ずる。本演習にはJavaを利用した。
\end{abstract}

\section{目的}
本稿は今年度発展プログラミングの課題研究として、「デザインパターン」に準拠したコーディングスタイルの必要性について論ずることを目的とする。本稿前半では、デザインパターンの利点・効果・必要性などについて説明し、後半では、実際にコーディングを行う。その際、デザインパターンを使用する前のコードとデザインパターンを使用した際のコードを比較・考察していく。
\section{デザインパターンとはなにか}
