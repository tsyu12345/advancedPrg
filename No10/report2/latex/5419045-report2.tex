\documentclass[dvipdfmx, titlepage]{jsarticle}
\usepackage[T1]{fontenc}
\usepackage[dvipdfmx]{hyperref}
\usepackage{lmodern}
\usepackage{latexsym}
\usepackage{amsfonts}
\usepackage{amssymb}
\usepackage{mathtools}
\usepackage{amsthm}
\usepackage{multirow}
\usepackage{graphicx}
\usepackage{wrapfig}
\usepackage{here}
\usepackage{float}
\usepackage{ascmac}
\usepackage{url}

\title{SpamCallIdentiferドキュメント}
\date{\today}
\author{日本大学 文理学部情報科学科\\5419045 高林 秀}

\begin{document}

\maketitle

\begin{abstract}
本ドキュメントは、今年度発展プログラミングの課題研究として作成する、Javaを使用したマルチスレッド処理を伴うアプリケーションの仕様を説明するものである。\par
本稿前半部では、本制作の目的、制作したアプリケーションの名前や仕様などの説明、実際に動作させる際の手順を解説する。後半部では、並行処理にまつわる歴史的な背景について軽く説明する。巻末部では本アプリケーションのソースコードを掲載したリポジトリのURLを示しているので、必要に応じてご参照いただきたい。\par
なお、本アプリケーションは開発環境と同等の環境のみ動作保証対象とする。
\end{abstract}

\section{目的}
本稿で紹介するアプリケーションは、今年度発展プログラミングの課題研究として、Javaを使用したマルチスレッド処理を伴うアプリケーションを開発することを目的とする。並びに、開発を通してJava、マルチスレッドプログラミングに対する理解を深めることを目的とする。\par
\section{前提知識}
本稿では「マルチスレッド」と呼ばれる言葉をしばしば「並行処理」と言い換えて表現する部分がある。\par
なお、Javaの言語特性、基本仕様、アプリケーションに関係のない部分に関する説明は本稿では省略する。\par
加えて、本アプリケーションの開発にはクラスや、継承などの機能を使用する。その部分に関する説明は以下のURLから、レポート「interface、抽象クラスを利用した Java のペア・プログラミング」を参考いただきたい。
\begin{itemize}
  \item interface、抽象クラスを利用した Java のペア・プログラミング:\url{https://drive.google.com/drive/folders/1QEt-NBptDGq2J1BgOyFnGUx8SMwL5oNc?usp=sharing}
\end{itemize}
\section{アプリケーション名}
SpamCallIdentifer
\subsection{名前の由来}
単純に「迷惑電話を判定するもの」という意味で名付けた。
\begin{itemize}
  \item Spam:「迷惑」という意。
  \item Call:「電話」。
  \item Identifer:「判定するもの」という意。
\end{itemize}
\section{アプリケーション概要}
SpamCallIdentiferは、入力として電話番号(ハイフン無し)を受け取り、出力としてその電話番号が「スパムである」か「スパムでない」かを判定した結果を表示する。この時、入出力はGUI表示で行う。
\section{想定シーン}
発信者が誰かわからない人物から電話がかかって来たとき、そのまま折返し電話をする前に使用することを想定している。\par
身元不明の人物に折返し電話をするのは非常に危険であり、最悪詐欺などの誘導に引っかかるケースも考えられる。そこで、SpamCallIdentiferに発信者の電話番号を入力し、怪しい発信者か否か判定することで、折返し電話をするべきか否かの判断の手助けをすることを目的とする。
\section{実装上の制約}
本アプリケーションは、入力された電話番号をインターネット上で検索し、その番号がスパムに該当するか否か調査する処理の実装を提出環境、開発環境の都合上行っていない。したがって下記のように代替の実装を行っている。
%ここにその説明を。
\begin{table}[H]
  \begin{tabular}{|l|l|}\hline
  本来の実装        & 代替の実装                       \\\hline
  番号の照会処理      & Thread.sleep()              \\\hline
  照会した電話番号のデータ & JudgeSpamクラス内の配列spamNumbers \\ \hline
  \end{tabular}
\end{table}
\section{クラス・メソッドの説明}
今回作成した各クラスをJavaファイルごとに説明する。
\begin{itemize}
  \item Main.java
  \begin{table}[H]
  \begin{tabular}{|l|l|} \hline
  メソッド名 & 説明                                                         \\ \hline
  main  & アプリケーションの呼び出しを行う。\\
  &SpamCallIdentiferクラスのdisplayメソッドを呼び出している。\\ \hline
  \end{tabular}
\end{table}
  \item JudgeSpam.java
  \begin{table}[H]
  \begin{tabular}{|l|l|} \hline
  メソッド名       & 説明                                                                                        \\ \hline
  run         & 番号照会を、同クラス内の配列spamNumbersでシミュレーションする。
  \\&spamNumbers内に入力された番号が含まれていないか判定する。 \\ \hline
  inputNumber & 引数:String
  \\&同クラス内のクラス変数numberへ引数の文字列を渡す。    \\ \hline
  \end{tabular}
\end{table}
  \item SpamCallIdentifer.java
  \begin{enumerate}
    \item  SpamCallIdentiferクラス
    \begin{table}[H]
\begin{tabular}{|l|l|} \hline
メソッド名            & 説明                                                                     \\ \hline
loadingAnimation & 番号照会中のローディングアニメーションのコンポーネントを作成し、返却する。\\&返り値:JPanel      \\ \hline
actionPerformed  & 描画されたボタンのクリックイベントをリスンし、クリック後の各処理を呼び出す。\\&引数:ActionEvent \\ \hline
display          & コンストラクタで生成した初期コンポーネントの配置と描画を行う。                                       \\ \hline
\end{tabular}
\end{table}
\item EndCatcherクラス
\begin{table}[H]
  \begin{tabular}{|l|l|} \hline
  メソッド名 & 説明                                                  \\ \hline
  run   & 照会処理(JudgeSpamクラス)のスレッドの終了を監視し、新規ウィンドウを生成し、結果を表示する。 \\ \hline
  \end{tabular}
\end{table}
  \end{enumerate}
\end{itemize}
\section{並行処理部分の概要}
今回は、以下2つの部分でマルチスレッドを実装している。
\begin{itemize}
  \item EndCatcherクラス
  \item JudgeSpam
\end{itemize}
マルチスレッドを実装するため、以上の2クラスにはRunnableインタフェースをimplementsしている。\par
JudgeSpamクラスは、インターネット上で入力された番号を検索するという、時間のかかる処理を想定している。また此のクラスのrunメソッドは、ユーザーが操作するGUI画面にある「判定開始」ボタンをクリックした時にThread.startメソッドを介して実行される。\par

\section{並行処理実現のための工夫した点}
先程も述べたが、インターネット上で番号検索を行う処理を想定しているJudgeSpamクラスは処理に時間のかかるクラスである。したがって、GUI表示を行うSpamCallIdentiferクラスと同一のスレッドで実行してしまうと、GUI表示の描画処理がJudgeSpamクラスのrunメソッドが終了するまで停止してしまい、画面が応答しなくなってしまう。\par
通常、GUIアプリケーションでは、画面を制御しているスレッドにて、数十ミリ秒以上時間がかかる処理を実行してはならない。なぜなら、画面の反応が悪く、OSがアプリケーションを応答停止状態と見なし動作を停止してしまうからである。これを回避するため、JudgeSpamクラスのrunメソッドをSpamCallIdentiferクラス内のスレッドとは別スレッドで呼び出している。このようにすることで、JudgeSpamクラスのrunメソッドの処理と、SpamCallIdentiferクラスの各メソッドが行うGUI描画更新処理を両方実行し、アプリケーションが応答停止になることを防止している。
\section{開発環境}
\begin{itemize}
  \item ホストOS:Window10 Home 20H2
  \item 仮想OS:Ubuntu 20.04.2 LTS
  \item CPU:Intel(R)Core(TM)i7-9700K @ 3.6GHz
  \item GPU:Nvidia Geforce RTX2070 OC @ 8GB
  \item ホストRAM:16GB
  \item 仮想RAM:4GB
  \item 使用エディタ:Microsoft Visual Studio Code
  \item 使用言語:Java
  \begin{itemize}
    \item バージョン情報は下記に示す。
    \begin{verbatim}
openjdk version "11.0.11" 2021-04-20
OpenJDK Runtime Environment (build 11.0.11+9-Ubuntu-0ubuntu2.20.04)
OpenJDK 64-Bit Server VM (build 11.0.11+9-Ubuntu-0ubuntu2.20.04, mixed mode, sharing)
    \end{verbatim}
    \item 利用したライブラリは下記。
    \begin{itemize}
      \item swing
    \end{itemize}
  \end{itemize}
\end{itemize}
\section{付録:並行処理の必要性}
そもそも何故マルチスレッド、マルチプロセスの機能が生まれたのかについて軽く説明する。\par
初期のコンピュータプログラムは、直接機械語でプログラムが書かれた磁気テープなどを読み取り、直接CPUに指示を出し実行するものであった。またCPUは基本的に計算装置を1つしか持たない「シングルプロセッサ」で1つずつ順番にしか処理することのできないものであった。一般に、ハードディスクなどの記憶装置に計算結果を書き込む場合は、CPUでの演算にかかる時間に比べ数十〜数百倍の時間がかかるとされる。したがって、1つ1つ順番にしか処理することのできないコンピュータ・プログラムはこれら長い処理が行われている間なにも処理する事ができない「待機時間」が発生することになる。マルチスレッド、マルチプロセスのCPU、プログラムはこれらの待機時間の合間に別の処理を行うことができるので、計算の効率を挙げることができる。\par
現代では、基本的にOS(オペレーティングシステム)と呼ばれるソフトウェアがそのコンピュータ上で実行される各アプリケーションの制御に関わっている。このOSと呼ばれるプログラムは、アプリケーションの制御だけでなく、次に実行されるプログラムの計算リソースの確保や、ハードウェアの管理などアプリケーションを実行する上で、様々なタスクも同時に行う必要がある。前述したシングルスレッド、シングルプロセスの場合では、2つ以上のプログラムを同時に実行することはできない。そのためマルチスレッド、マルチプロセスといった並行して処理を行うような仕組みが必要とされた。
\section{巻末付録}
アプリケーションのソースコードは以下のリポジトリに掲載している。
\begin{itemize}
  \item GitHub:\url{https://github.com/tsyu12345/advancedPrg}
\end{itemize}

\end{document}
