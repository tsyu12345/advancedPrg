\documentclass[dvipdfmx]{jsarticle}
\usepackage[T1]{fontenc}
\usepackage[dvipdfmx]{hyperref}
\usepackage{lmodern}
\usepackage{latexsym}
\usepackage{amsfonts}
\usepackage{amssymb}
\usepackage{mathtools}
\usepackage{amsthm}
\usepackage{multirow}
\usepackage{graphicx}
\usepackage{wrapfig}
\usepackage{here}
\usepackage{float}
\usepackage{ascmac}
\usepackage{url}

\title{ネットワークプログラミングを利用したアプリケーション開発}
\date{\today}
\author{日本大学 文理学部情報科学科\\5419045 高林 秀}

\begin{document}

\maketitle

\begin{abstract}
本稿は、今年度発展プログラミングの課題研究としてProcessingを用いたネットワークプログラミングを使用したアプリケーション開発を行うものである。本稿前半部では、開発の際に利用した技術やコードに関して説明を行う。本稿後半部では、実際に説明した技術を用いて、Processing上で実行可能なアプリケーションを作成する。
\end{abstract}

\tableofcontents

\section{目的}
本稿の目的は、今年度発展プログラミングの課題研究としてProcessingを用いたネットワークプログラミングを使用したアプリケーション開発を行うものである。前半部にて基本的な技術用語の説明を通して、プログラミングの際に必要な知識の復習を行う。後半部では、実際にProcessing上で実行可能なアプリケーション開発を通してネットワークプログラミングを自身のプログラムに実装する。
\section{Processing上でのネットワークプログラミング概要}
\section{クライアントサイドプログラミング}
\section{サーバーサイドプログラミング}
\section{アプリケーション実装}
\subsection{開発環境}
今回の開発は仮想マシン上で行った。下記に当時の環境を示す。
\begin{itemize}
  \item ホストOS:Window10 Home 20H2
  \item 仮想OS:Ubuntu 20.04.2 LTS
  \item CPU:Intel(R)Core(TM)i7-9700K @ 3.6GHz
  \item GPU:Nvidia Geforce RTX2070 OC @ 8GB
  \item ホストRAM:16GB
  \item 仮想RAM:4GB
  \item Processing version : 3.5.3
\end{itemize}
\subsection{制作内容}
\subsection{挙動説明}
\subsection{工夫点}
\section{巻末資料}
本稿で使用した画像、プログラムコード等はすべて以下のリンク先に掲載している。必要に応じてご覧頂きたい。
\begin{itemize}
  \item GoogleDrive:\url{}
  \item GitHub:\url{}
\end{itemize}




\end{document}
