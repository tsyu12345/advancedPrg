\documentclass[dvipdfmx]{jsarticle}
\usepackage[T1]{fontenc}
\usepackage[dvipdfmx]{hyperref}
\usepackage{lmodern}
\usepackage{latexsym}
\usepackage{amsfonts}
\usepackage{amssymb}
\usepackage{mathtools}
\usepackage{amsthm}
\usepackage{multirow}
\usepackage{graphicx}
\usepackage{wrapfig}
\usepackage{here}
\usepackage{float}
\usepackage{ascmac}
\usepackage{url}

\title{Javaを用いたデザインパターンコーディングの必要性}
\author{文理学部情報科学科\\5419045 高林 秀}
\date{\today}

\begin{document}

\maketitle

\begin{abstract}
本稿では、今年度発展プログラミングの課題研究として「デザインパターン」に準拠したコーディングスタイルの必要性について、実際に自身でコーディングを行い、論ずる。本演習にはJavaを利用した。
\end{abstract}

\section{目的}
本稿は今年度発展プログラミングの課題研究として、「デザインパターン」に準拠したコーディングスタイルの必要性について論ずることを目的とする。本稿前半では、デザインパターンの利点・効果・必要性などについて説明し、後半では、実際にコーディングを行う。その際、デザインパターンを使用する前のコードとデザインパターンを使用した際のコードを比較・考察していく。
\section{デザインパターンとはなにか}
この章では、プログラムコーディングおけるデザインパターンとはどのようなものなのか、歴史的背景や利点・効果、その必要性について述べる。\par
\subsection{歴史的背景}
デザインパターンは、オブジェクト指向型プログラミング言語において、記述したコードを様々なプログラムで再利用できるようにするために考案された「プログラムの設計ルール」のようなものである。1955年に出版された書籍「オブジェクト指向における再利用のためのデザインパターン\cite{book01}」にて、初めて「デザインパターン」と呼ばれる用語が使用された。その書籍の広がりにより、デザインパターンの考え方が広く知られるようになった。\par
その書籍の著者ら(※参考文献の原著4名)は、23種にもおよぶデザインパターンを取り上げており、デザインパターンとはなにか、以下のように述べている。以下、\url{https://ja.wikipedia.org/wiki/%E3%83%87%E3%82%B6%E3%82%A4%E3%83%B3%E3%83%91%E3%82%BF%E3%83%BC%E3%83%B3_(%E3%82%BD%E3%83%95%E3%83%88%E3%82%A6%E3%82%A7%E3%82%A2)}より引用する。
\begin{itemize}
  \item 原文
  \begin{quote}
    [Design patterns] solve specific design problems and make object-oriented designs more flexible, elegant, and ultimately reusable. They help designers reuse successful designs by basing new designs on prior experience. A designer who is familiar with such patterns can apply them immediately to design problems without having to rediscover them.
  \end{quote}
  \item 訳文(Deepl\footnote{ドイツに拠点を置くDeepL GmbHによって開発された、ニューラルネットワークによる翻訳を行うサービス。Google 翻訳よりも精度が高く、微妙なニュアンスのある翻訳ができると話題。}による翻訳)
  \begin{quote}
    特定の設計上の問題を解決し、オブジェクト指向設計をより柔軟に、エレガントに、そして最終的には再利用可能にします。デザインパターンは、新しいデザインを過去の経験に基づいて行うことで、成功したデザインを再利用するのに役立ちます。このようなパターンに精通している設計者は、パターンを再発見することなく、すぐに設計問題に適用することができます。
  \end{quote}
\end{itemize}
この書籍の著者たちは「Gof(Gang of Four)」と呼ばれデザインパターンの名前の1つにもなっている。\par
Gofのデザインパターンはかなり前のデザインパターンである。そのため、現代では賛否両論あるようで、一部では批判的な意見も挙げられている。というのも、後に説明するが、デザインパターンはJavaやRubyなどオブジェクト指向型言語で使用される考え方だ。しかしGofのデザインパターンが発表されたのはJavaがリリースされる1995年よりも前、つまりオブジェクト指向プログラミングというものが未熟な時代に発表されたからだ。したがって、現代のプログラミングと合致しない場合もあるという。この意見は\url{https://qiita.com/irxground/items/d1f9cc447bafa8db2388}より参考にさせていただいた。
\subsection{デザインパターンの利点}
前述した通り、デザインパターンの主目的は「コードを様々なプログラムで再利用できるようにする」ことである。ただし、利用するデザインパターンにより、その効果は少し異なる。今回扱う、Gofのデザインパターンは大きく分けて、「生成」、「構造」,「振る舞い」の3つのカテゴリに分類され、その総数は後述するように計23パターンにおよぶ。\par
以上のように分類され、体系が出来上がった概念、考え方なので巷では初心者エンジニアが、オブジェクト指向を学習するときの題材として利用しやすいことも1つの利点と言えるだろう。\par
さて、デザインパターンの大きな利点として、先に上げた「コードの再利用」の他に以下の3つが挙げられるだろう。
\begin{itemize}
  \item 可読性の向上
  \item 保守性の向上
  \item 設計の短時間化
\end{itemize}
\paragraph{可読性の向上}まず、「可読性の向上」だがこれは自分が作成したプログラムを、他のエンジニアやプログラマーが閲覧するときに、そのプログラム、クラスがどんな役割を果たしているのか、どのような意図のコードなのかが把握しやすくなるという意味だ。先に示したように、デザインパターンはすでに確立されたプログラムの設計概念である。したがって、いわば共通言語のようなものであるので、デザインパターンを意識したプログラムは他人からも読みやすいということになる。\par
\paragraph{保守性の向上}次に、「保守性の向上」だがこれは、例えば呼び出し元プログラムの実装方法が変更になった場合でも呼び出し側のプログラムに修正を加えることなく実行できるということだ。大規模な開発になると、実装元と呼び出し側にコードを分散させることが多々あるが、仕様変更やアップデート等で実装側のコードを改変した場合、デザインパターンを意識しないで呼び出し側を設計すると、実装元と呼び出し側両方のプログラムを修正しなければならない。この様なプログラムは、実装側の修正箇所が多くなるたびに呼び出し側も修正しなければならないため非常に効率が悪い。加えて、実装側のコードや動作を把握しなければ呼び出し側の設計ができないという欠点も生じる。\par
デザインパターンを導入すれば、少ない修正でプログラムを動かすことが可能になる。
\paragraph{設計の短時間化}最後に「設計の短時間化」であるが、これはコードの再利用の点と一部重なるかもしれない。デザインパターンは、Gofのデザインパターンの著者を初め、すでに誰かが考え出した設計にしたがってプログラムの設計を行う。したがって、自身で保守性や再利用性を向上させようとあれこれ試行錯誤する必要はなく、コーディング作業全体の効率化を図ることができる。また、先人のアイデアを得ることもできるだろう。
\subsubsection{デザインパターンの必要性}
ここまで、デザインパターンの利点を述べてきたが、ここではデザインパターンの利用が推奨されるケースについて説明する。
\subsection{デザインパターンの種類}


\section{実際のコーディングでデザインパターンを利用する}
\subsection{課題説明}
\paragraph{問題} 任意のデザインパターンを1つ選択し、そのパターンを用いる前と用いた後のコード両方を作成して示し、そのパターンを用いたことの効果を説明せよ。
\begin{itemize}
  \item 選択したデザインパターン:
\end{itemize}

\subsubsection{演習環境}
今回の演習は仮想マシン上でJavaを使用し行った。下記に演習時の環境を示す。
\begin{itemize}
  \item ホストOS:Window10 Home 20H2
  \item 仮想OS:Ubuntu 20.04.2 LTS
  \item CPU:Intel(R)Core(TM)i7-9700K @ 3.6GHz
  \item GPU:Nvidia Geforce RTX2070 OC @ 8GB
  \item ホストRAM:16GB
  \item 仮想RAM:4GB
  \item 使用言語:Java
  \begin{itemize}
    \item バージョン情報は下記に示す。
    \begin{verbatim}
      openjdk version "11.0.11" 2021-04-20
      OpenJDK Runtime Environment (build 11.0.11+9-Ubuntu-0ubuntu2.20.04)
      OpenJDK 64-Bit Server VM (build 11.0.11+9-Ubuntu-0ubuntu2.20.04, mixed mode, sharing)
    \end{verbatim}
  \end{itemize}
\end{itemize}

\subsection{制作物}
\subsubsection{デザインパターン前のソースコード}
\subsubsection{デザインパターン後のソースコード}
\subsection{考察}
\section{まとめ}
\section{巻末資料}
\begin{thebibliography}{99}
  \bibitem{book01}  Erich Gamma (原著), Ralph Johnson (原著), Richard Helm (原著), John Vlissides (原著), 本位田 真一 (翻訳), 吉田 和樹 (翻訳)「オブジェクト指向における再利用のためのデザインパターン」改訂版(ソフトバンククリエイティブ,1999/10/1)
\end{thebibliography}

\end{document}
